\documentclass{ximera}

%\usepackage{todonotes}

\newcommand{\todo}{}

\usepackage{tkz-euclide}
\tikzset{>=stealth} %% cool arrow head
\tikzset{shorten <>/.style={ shorten >=#1, shorten <=#1 } } %% allows shorter vectors

\usetikzlibrary{backgrounds} %% for boxes around graphs
\usetikzlibrary{shapes,positioning}  %% Clouds and stars
\usetikzlibrary{matrix} %% for matrix
\usepgfplotslibrary{polar} %% for polar plots
\usetkzobj{all}
\usepackage[makeroom]{cancel} %% for strike outs
%\usepackage{mathtools} %% for pretty underbrace % Breaks Ximera
\usepackage{multicol}





\usepackage{array}
\setlength{\extrarowheight}{+.1cm}   
\newdimen\digitwidth
\settowidth\digitwidth{9}
\def\divrule#1#2{
\noalign{\moveright#1\digitwidth
\vbox{\hrule width#2\digitwidth}}}





\newcommand{\RR}{\mathbb R}
\newcommand{\R}{\mathbb R}
\newcommand{\N}{\mathbb N}
\newcommand{\Z}{\mathbb Z}

%\renewcommand{\d}{\,d\!}
\renewcommand{\d}{\mathop{}\!d}
\newcommand{\dd}[2][]{\frac{\d #1}{\d #2}}
\newcommand{\pp}[2][]{\frac{\partial #1}{\partial #2}}
\renewcommand{\l}{\ell}
\newcommand{\ddx}{\frac{d}{\d x}}

\newcommand{\zeroOverZero}{\ensuremath{\boldsymbol{\tfrac{0}{0}}}}
\newcommand{\inftyOverInfty}{\ensuremath{\boldsymbol{\tfrac{\infty}{\infty}}}}
\newcommand{\zeroOverInfty}{\ensuremath{\boldsymbol{\tfrac{0}{\infty}}}}
\newcommand{\zeroTimesInfty}{\ensuremath{\small\boldsymbol{0\cdot \infty}}}
\newcommand{\inftyMinusInfty}{\ensuremath{\small\boldsymbol{\infty - \infty}}}
\newcommand{\oneToInfty}{\ensuremath{\boldsymbol{1^\infty}}}
\newcommand{\zeroToZero}{\ensuremath{\boldsymbol{0^0}}}
\newcommand{\inftyToZero}{\ensuremath{\boldsymbol{\infty^0}}}



\newcommand{\numOverZero}{\ensuremath{\boldsymbol{\tfrac{\#}{0}}}}
\newcommand{\dfn}{\textbf}
%\newcommand{\unit}{\,\mathrm}
\newcommand{\unit}{\mathop{}\!\mathrm}
\newcommand{\eval}[1]{\bigg[ #1 \bigg]}
\newcommand{\seq}[1]{\left( #1 \right)}
\renewcommand{\epsilon}{\varepsilon}
\renewcommand{\iff}{\Leftrightarrow}

\DeclareMathOperator{\arccot}{arccot}
\DeclareMathOperator{\arcsec}{arcsec}
\DeclareMathOperator{\arccsc}{arccsc}
\DeclareMathOperator{\si}{Si}
\DeclareMathOperator{\proj}{proj}
\DeclareMathOperator{\scal}{scal}


\newcommand{\tightoverset}[2]{% for arrow vec
  \mathop{#2}\limits^{\vbox to -.5ex{\kern-0.75ex\hbox{$#1$}\vss}}}
\newcommand{\arrowvec}[1]{\tightoverset{\scriptstyle\rightharpoonup}{#1}}
\renewcommand{\vec}{\mathbf}
\newcommand{\veci}{\vec{i}}
\newcommand{\vecj}{\vec{j}}
\newcommand{\veck}{\vec{k}}
\newcommand{\vecl}{\boldsymbol{\l}}

\newcommand{\dotp}{\bullet}
\newcommand{\cross}{\boldsymbol\times}
\newcommand{\grad}{\boldsymbol\nabla}
\newcommand{\divergence}{\grad\dotp}
\newcommand{\curl}{\grad\cross}
%\DeclareMathOperator{\divergence}{divergence}
%\DeclareMathOperator{\curl}[1]{\grad\cross #1}


\colorlet{textColor}{black} 
\colorlet{background}{white}
\colorlet{penColor}{blue!50!black} % Color of a curve in a plot
\colorlet{penColor2}{red!50!black}% Color of a curve in a plot
\colorlet{penColor3}{red!50!blue} % Color of a curve in a plot
\colorlet{penColor4}{green!50!black} % Color of a curve in a plot
\colorlet{penColor5}{orange!80!black} % Color of a curve in a plot
\colorlet{fill1}{penColor!20} % Color of fill in a plot
\colorlet{fill2}{penColor2!20} % Color of fill in a plot
\colorlet{fillp}{fill1} % Color of positive area
\colorlet{filln}{penColor2!20} % Color of negative area
\colorlet{fill3}{penColor3!20} % Fill
\colorlet{fill4}{penColor4!20} % Fill
\colorlet{fill5}{penColor5!20} % Fill
\colorlet{gridColor}{gray!50} % Color of grid in a plot

\newcommand{\surfaceColor}{violet}
\newcommand{\surfaceColorTwo}{redyellow}
\newcommand{\sliceColor}{greenyellow}




\pgfmathdeclarefunction{gauss}{2}{% gives gaussian
  \pgfmathparse{1/(#2*sqrt(2*pi))*exp(-((x-#1)^2)/(2*#2^2))}%
}


%%%%%%%%%%%%%
%% Vectors
%%%%%%%%%%%%%

%% Simple horiz vectors
\renewcommand{\vector}[1]{\left\langle #1\right\rangle}


%% %% Complex Horiz Vectors with angle brackets
%% \makeatletter
%% \renewcommand{\vector}[2][ , ]{\left\langle%
%%   \def\nextitem{\def\nextitem{#1}}%
%%   \@for \el:=#2\do{\nextitem\el}\right\rangle%
%% }
%% \makeatother

%% %% Vertical Vectors
%% \def\vector#1{\begin{bmatrix}\vecListA#1,,\end{bmatrix}}
%% \def\vecListA#1,{\if,#1,\else #1\cr \expandafter \vecListA \fi}

%%%%%%%%%%%%%
%% End of vectors
%%%%%%%%%%%%%

%\newcommand{\fullwidth}{}
%\newcommand{\normalwidth}{}



%% makes a snazzy t-chart for evaluating functions
%\newenvironment{tchart}{\rowcolors{2}{}{background!90!textColor}\array}{\endarray}

%%This is to help with formatting on future title pages.
\newenvironment{sectionOutcomes}{}{} 



%% Flowchart stuff
%\tikzstyle{startstop} = [rectangle, rounded corners, minimum width=3cm, minimum height=1cm,text centered, draw=black]
%\tikzstyle{question} = [rectangle, minimum width=3cm, minimum height=1cm, text centered, draw=black]
%\tikzstyle{decision} = [trapezium, trapezium left angle=70, trapezium right angle=110, minimum width=3cm, minimum height=1cm, text centered, draw=black]
%\tikzstyle{question} = [rectangle, rounded corners, minimum width=3cm, minimum height=1cm,text centered, draw=black]
%\tikzstyle{process} = [rectangle, minimum width=3cm, minimum height=1cm, text centered, draw=black]
%\tikzstyle{decision} = [trapezium, trapezium left angle=70, trapezium right angle=110, minimum width=3cm, minimum height=1cm, text centered, draw=black]



\outcome{Find the equations for planes.}
\outcome{Give the equation for a sphere or ball.}

\title[Dig-In:]{Two and three dimensional geometry}

\begin{document}
\begin{abstract}
  We talk about basic geometry in higher dimensions.
\end{abstract}
\maketitle



In three dimensions we have three coordinates axes, the $x$-axis,
$y$-axis, and $z$-axis:
\begin{image}[1in]
\begin{tikzpicture}
  \draw[->] (0,0)--(1,0);
  \draw[->] (0,0)--(0,1);
  \draw[->] (0,0)--(-.4,-.7);
  \node[right] at (1,0) {$x$};
  \node[above] at (0,1) {$y$};
  \node[below] at (-.4,-.7) {$z$};
  
\end{tikzpicture}
\end{image}
The axes point according to the \dfn{right-hand-rule}, though you might
need to ``spin'' your hand around:
\begin{image}[2in]
  %% Based on: https://commons.wikimedia.org/wiki/File:Right_hand_rule_cross_product.svg
  %% Which was based on: https://commons.wikimedia.org/wiki/File:Right_hand_cross_product.png
  %% Which was based on: https://commons.wikimedia.org/wiki/File:LeftHandOutline.png
  \begin{tikzpicture}[y=0.80pt, x=0.80pt, yscale=-1.000000, xscale=1.000000, inner sep=0pt, outer sep=0pt]
    \path[draw=black,fill=blue!50!white,line width=2.400pt] (466.9820,303.7880) ..
    controls (444.9820,300.7880) and (409.9820,280.7880) .. (404.9820,277.7880) ..
  controls (399.9820,274.7880) and (376.9820,249.7880) .. (364.9820,230.7880) ..
  controls (352.9820,211.7880) and (341.9820,188.7880) .. (337.9820,175.7880) ..
  controls (333.9820,162.7880) and (329.9820,149.7880) .. (328.9820,142.7880) ..
  controls (327.9820,135.7880) and (329.9820,119.2880) .. (332.9820,112.2880) ..
  controls (325.9820,107.2880) and (311.9820,106.2880) .. (305.9820,119.2880) ..
  controls (299.9820,132.2880) and (294.9820,152.2880) .. (294.9820,165.2880) ..
  controls (294.9820,178.2880) and (316.9820,220.2880) .. (302.9820,226.2880) ..
  controls (288.9820,232.2880) and (285.9820,240.2880) .. (213.9820,222.2880) ..
  controls (141.9820,204.2880) and (111.9820,202.2880) .. (104.9820,216.2880) ..
  controls (97.9820,230.2880) and (138.9820,236.2880) .. (160.9820,240.2880) ..
  controls (182.9820,244.2880) and (233.0030,262.9220) .. (225.9820,272.2880) ..
  controls (221.6320,278.0900) and (216.6070,276.7760) .. (205.2730,280.7760) ..
  controls (185.6620,287.6980) and (185.9650,290.7420) .. (161.4820,299.1210) ..
  controls (137.0490,307.4830) and (138.6390,331.3470) .. (146.3150,335.3510) ..
  controls (153.9910,339.3550) and (160.3310,341.6930) .. (192.7010,323.3380) ..
  controls (207.0510,317.9980) and (224.4830,314.4530) .. (239.8160,313.1200) ..
  controls (255.1490,311.7870) and (264.4840,369.1200) .. (281.1500,377.7870) ..
  controls (297.8160,386.4540) and (317.1490,385.7870) .. (329.1490,385.7870) ..
  controls (341.1490,385.7870) and (347.8150,387.7880) .. (354.4820,384.4540) ..
  controls (361.1490,381.1200) and (379.8160,385.7860) .. (389.8160,391.1200) ..
  controls (389.8160,391.1200) and (428.4830,409.7870) .. (431.8160,413.7870);
\path[draw=black,line width=1.600pt] (234.4830,270.7870) .. controls
  (268.4820,268.7880) and (280.4820,301.4540) .. (288.4820,318.7880);
\path[draw=black,line width=1.600pt] (306.1490,245.4530) .. controls
  (306.4830,270.7870) and (317.1500,292.1210) .. (331.4820,300.1190);
\path[draw=black,line width=1.600pt] (239.8160,230.7870) .. controls
  (231.8160,232.1200) and (222.4830,250.7870) .. (226.4830,256.1200);
\path[draw=black,line width=1.600pt] (197.1490,220.1200) .. controls
  (191.8160,224.1200) and (181.1490,233.4530) .. (183.8160,238.7870);
\path[draw=black,line width=1.600pt] (149.1490,212.1200) .. controls
  (142.4820,216.1200) and (140.3150,227.6210) .. (138.9820,230.2870);
\path[draw=black,line width=1.600pt] (222.4830,280.1210) .. controls
  (230.4830,284.1210) and (234.4830,297.4530) .. (231.8160,306.7870);
\path[draw=black,line width=1.600pt] (201.2960,287.3930) .. controls
  (207.9630,291.3930) and (208.5440,311.3860) .. (205.8770,315.3860);
\path[draw=black,line width=1.600pt] (175.8480,298.5900) .. controls
  (184.4960,305.0750) and (185.4390,314.1240) .. (180.9370,325.0560);
\path[draw=black,line width=1.600pt] (293.9820,236.2880) .. controls
  (289.9820,246.2880) and (286.9820,254.2880) .. (287.9820,260.2880);
\path[draw=black,line width=1.600pt] (294.9820,165.2880) .. controls
  (306.9290,166.9950) and (297.6590,168.5400) .. (319.9820,164.2880);
\path[draw=black,line width=1.600pt] (399.1490,286.7880) .. controls
  (395.1490,298.7880) and (391.1490,302.7870) .. (384.4820,306.7870);
\path[draw=black,fill=blue!50!white,line width=2.400pt] (299.8160,374.4550) ..
  controls (315.7210,371.2740) and (327.1500,369.1220) .. (335.8160,366.4550) ..
  controls (344.4820,363.7880) and (357.1480,356.4540) .. (360.4820,349.7880) ..
  controls (363.8160,343.1220) and (361.1490,341.7890) .. (359.1490,337.1220) ..
  controls (357.1490,332.4550) and (335.1500,335.1220) .. (329.8160,337.1220) ..
  controls (324.4820,339.1220) and (319.8160,341.1220) .. (297.8160,344.4550) ..
  controls (275.8160,347.7880) and (269.8160,349.4550) .. (258.4820,348.1210) ..
  controls (249.1900,347.0270) and (259.8170,370.4540) .. (264.4830,373.1210) ..
  controls (269.1490,375.7880) and (269.8160,380.4550) .. (299.8160,374.4550) --
  cycle;
\path[draw=black,fill=blue!50!white,line width=2.400pt] (316.7840,311.7880) ..
  controls (335.6810,310.4550) and (345.1320,309.1220) .. (350.2190,314.4550) ..
  controls (355.3060,319.7880) and (353.1250,325.1220) .. (351.6720,326.4550) ..
  controls (350.2190,327.7880) and (333.5010,335.1210) .. (324.7790,337.7880) ..
  controls (316.0560,340.4550) and (314.6030,338.4550) .. (297.1600,342.4550) ..
  controls (279.7160,346.4550) and (278.2630,347.7880) .. (273.1740,349.7880) ..
  controls (268.0860,351.7880) and (257.9110,351.1210) .. (248.4630,343.7880) ..
  controls (239.0140,336.4550) and (240.4670,323.7880) .. (241.1940,319.7880) ..
  controls (241.9210,315.7880) and (242.6490,307.1210) .. (260.8190,305.7880) ..
  controls (278.9890,304.4550) and (273.1730,310.4540) .. (296.4330,311.1210) ..
  controls (319.6910,311.7880) and (316.7840,311.7880) .. (316.7840,311.7880) --
  cycle;
\path[draw=black,line width=1.600pt] (245.8150,318.4550) .. controls
  (243.8920,335.7600) and (246.9240,336.6210) .. (256.4820,343.7880);
\path[draw=black,line width=1.600pt] (275.1490,318.7870) .. controls
  (275.1490,324.1200) and (276.4820,336.1200) .. (276.4820,336.1200);
\path[draw=black,line width=1.600pt] (308.1490,317.1200) .. controls
  (308.1490,319.7870) and (309.4830,330.4530) .. (309.4830,334.4530);
\path[draw=black,line width=1.600pt] (314.8150,347.1220) .. controls
  (314.8150,351.1220) and (321.4820,364.4550) .. (321.4820,364.4550);
\path[draw=black,line width=1.600pt] (282.4820,354.7880) .. controls
  (283.8150,357.4550) and (285.1490,369.4540) .. (289.1490,370.7880);
\path[draw=black,line width=1.600pt] (405.8150,318.7880) .. controls
  (399.1490,330.7880) and (395.1490,361.4560) .. (388.4820,368.1220);
\path[draw=black,fill=black,line cap=round,line width=7.200pt]
  (323.4820,90.7880) -- (322.8140,41.7880);
\path[fill=black] (322.8140,41.7880) -- (299.5010,60.3000) -- (322.8140,0.0000)
  -- (346.1270,60.3000) -- cycle;
\path[draw=black,fill=black,line cap=round,line width=7.200pt]
  (88.9960,210.4600) -- (40.9000,201.0670);
\path[fill=black] (40.9000,201.0670) -- (54.2400,227.6800) -- (0.0000,192.5000)
  -- (63.7990,182.0440) -- cycle;
\path[draw=black,fill=black,line cap=round,line width=7.200pt]
  (126.3770,335.2550) -- (95.5850,348.0050);
\path[fill=black] (95.5850,348.0050) -- (118.0640,371.4130) --
  (69.0630,358.1970) -- (96.6050,315.5680) -- cycle;
\path[fill=black,line join=miter,line cap=butt,line width=0.800pt]
  (51.2201,166.8365) node[above right] (text4204) {\scalebox{4}{$x$}};
\path[fill=black,line join=miter,line cap=butt,line width=0.800pt]
  (89.7117,303.4817) node[above right] (text4208) {\scalebox{4}{$y$}};
\path[fill=black,line join=miter,line cap=butt,line width=0.800pt]
  (364.9268,59.0600) node[above right] (text4212) {\scalebox{4}{$z$}};
\end{tikzpicture}
\end{image}

We can clearly see the $(x,y)$-plane in our axes, but there are also
two others: $(x,z)$-plane, and the $(y,z)$-plane.

\begin{question}
	The $(y,z)$-plane corresponds to which of the following
        equations?
	\begin{multipleChoice}
	  \choice[correct]{$x=0$}
	  \choice{$y=0$}
	  \choice{$z=0$}
	\end{multipleChoice}
	\begin{hint}
	  Every point on the $(y,z)$-axis has $x=0$, so this is the answer.
	\end{hint}
\end{question}

\begin{question}
  Which of the following most accurately describes the solution
  set of $y=2$ in $\R^3$?
  \begin{multipleChoice}
    \choice{a plane parallel to the $(x,y)$-plane}
    \choice[correct]{a plane parallel to the $(x,z)$-plane}
    \choice{a plane parallel to the $(y,z)$-plane}
  \end{multipleChoice}
  \begin{hint}
    $y=2$ consists of all those points where $y=2$, but $x$ and
    $z$ are allowed to be anything.  
  \end{hint}
\end{question}


\subsection{Lines}

Parametric formulas for lines that start at a point when $t=0$ and
progress in a direction given by a vector can be described in
\textit{any} dimension as
\[
\l(t) = \vec{point} + t\cdot\vec{vector}
\]
\begin{question}
  Give a parametric formula $\l(t)$ for a line such that
  \begin{itemize}
  \item $\l(0)=(1,2,3)$,
  \item $\l(1) = (-1,2,-3)$,
  \item $\l(2) = (-5,2,-9)$.
  \end{itemize}
  \begin{hint}
    What is the vector whose tip is at $(-1,2-3)$ and whose tail is at
    $(1,2,3)$?
    \begin{prompt}
    \[
    \vec{vector} = \vector{\answer{-2},\answer{0},\answer{-6}}
    \]
    \end{prompt}
  \end{hint}
  \begin{prompt}
    \[
    \l(t) = \vector{\answer{1},\answer{2},\answer{3}} + t\cdot \vector{\answer{-2},\answer{0},\answer{-6}}
    \]
  \end{prompt}
\end{question}

\subsection{Circles and spheres}

Just as the equation of a circle is related to the Pythagorean
theorem, the equation of a sphere is directly related to how the
lengths of vectors are computed.

\begin{theorem}
  The equation for a circle of radius $r$ centered at the point
  $(a,b)$ in $\R^2$ is
  \[
  r^2=(x-a)^2 + (y-b)^2
  \]
  The equation for a sphere of radius $r$ centered at the point
  $(a,b,c)$ in $\R^3$ is
  \[
  r^2 = (x-a)^2 + (y-b)^2 + (x-c)^2
  \]
  In general, the equation for a $n$-dimensional ``sphere'' of radius
  $r$ centered at the point $(a_1,a_2,a_3,\dots,a_n)$ is
  \[
  r^2 = \sum_{i=1}^n(x-a_i)^2
  \]
  \begin{explanation}
    This follows directly from the definition of the length of a
    vector, since a sphere is nothing more than the set of points that
    are equidistant from a given point.
  \end{explanation}
\end{theorem}

\begin{corollary}
  In general, the equation for a $n$-dimensional ``ball'' of radius
  $r$ centered at the point $(a_1,a_2,a_3,\dots,a_n)$ is
  \[
  r^2 \ge \sum_{i=1}^n(x-a_i)^2
  \]
  \begin{explanation}
    Here the ``$\ge$'' fills-in the $n$-dimensional sphere.
  \end{explanation}
\end{corollary}


\begin{question}
  The equation $(x-1)^2+y^2+(z+2)^2 = 4$ has a solution set in
  $\R^3$ which is a sphere.  What is the center and
  radius of this sphere?
  \begin{prompt}
  \[
  \text{Radius} = \answer{2}
  \]
  \[
  \text{Center} = \left(\answer{1},\answer{0}, \answer{-2} \right)
  \]
  \end{prompt}
  \begin{hint}
    $(x-1)^2+y^2+(z+2)^2$ is the square of the distance from
    $(1,0,-2)$ to $(x,y,z)$.  If the square of the distance is $4$, then
    the distance is $2$.  Since the solution set of this equation is all
    points which are a distance of $2$ away from $(1,0,-2)$, then this
    is a sphere of radius $2$ centered at $(1,0,-2)$
  \end{hint}
\end{question}

\end{document}
